\documentclass{article}

\usepackage{tabularx}
\usepackage{booktabs}

\title{SE 3XA3: Problem Statement\\Title of Project}

\author{Team \# 3, 3
		\\ Erin Varey-vareye
		\\ Joel Straatmen-straatjc
		\\ Nik Novak-novakn
}

\date{}

%% Comments

\usepackage{color}

\newif\ifcomments\commentstrue

\ifcomments
\newcommand{\authornote}[3]{\textcolor{#1}{[#3 ---#2]}}
\newcommand{\todo}[1]{\textcolor{red}{[TODO: #1]}}
\else
\newcommand{\authornote}[3]{}
\newcommand{\todo}[1]{}
\fi

\newcommand{\wss}[1]{\authornote{blue}{SS}{#1}}
\newcommand{\ds}[1]{\authornote{red}{DS}{#1}}
\newcommand{\mj}[1]{\authornote{red}{MSN}{#1}}
\newcommand{\cm}[1]{\authornote{red}{CM}{#1}}
\newcommand{\mh}[1]{\authornote{red}{MH}{#1}}

% team members should be added for each team, like the following
% all comments left by the TAs or the instructor should be addressed
% by a corresponding comment from the Team

\newcommand{\tm}[1]{\authornote{magenta}{Team}{#1}}


\begin{document}

\begin{table}[hp]
\caption{Revision History} \label{TblRevisionHistory}
\begin{tabularx}{\textwidth}{llX}
\toprule
\textbf{Date} & \textbf{Developer(s)} & \textbf{Change}\\
\midrule
December 5th, 2016 & Erin Varey & Final Modifications\\
... & ... & ...\\
\bottomrule
\end{tabularx}
\end{table}

\newpage

\maketitle

	It is difficult for a social media user to avoid certain topics. If a person wants to avoid particular online material there is the block option, but this removes everything from that user rather than the topic itself. Rather is a web application that allows users to filter images and topics they do not want to see on Facebook and replace it with images they do want to see. Our intended user is anyone who uses the social media website Facebook and wants to filter what they see on their feed. \\
The filtering of what you see online is really important for anyone but in particular it’s important for sensitive users. If someone suffers from a condition like PTSD social media can trigger these horrific memories. With the use of Rather the user has the choice to filter away not safe for work imagery and replace it with more appropriate images. This makes it even easier for the user to avoid things they do not want to see and replace them with friendlier images.  This software is intended to be used on personal computers that run Google Chrome. 


\end{document}