\documentclass[12pt, titlepage]{article}
\usepackage{fullpage}
\usepackage[round]{natbib}
\usepackage{multirow}
\usepackage{booktabs}
\usepackage{tabularx}
\usepackage{graphicx}
\usepackage{float}
\usepackage{hyperref}
\hypersetup{
    colorlinks,
    citecolor=black,
    filecolor=black,
    linkcolor=red,
    urlcolor=blue
}
\usepackage[round]{natbib}
\newcounter{acnum}
\newcommand{\actheacnum}{AC\theacnum}
\newcommand{\acref}[1]{AC\ref{#1}}
\newcounter{ucnum}
\newcommand{\uctheucnum}{UC\theucnum}
\newcommand{\uref}[1]{UC\ref{#1}}
\newcounter{mnum}
\newcommand{\mthemnum}{M\themnum}
\newcommand{\mref}[1]{M\ref{#1}}
\title{SE 3XA3: Software Requirements Specification\\Rather 2.0}
\author{Team \#, Team 3
		\\ Erin Varey, Vareye
		\\ Nik Novak, Novakn
		\\ Joel Straatman, Straatjc
}
\date{\today}
\begin{document}
\maketitle
\pagenumbering{roman}
\tableofcontents
\listoftables
\listoffigures
\begin{table}[bp]
\caption{\bf Revision History}
\begin{tabularx}{\textwidth}{p{3cm}p{2cm}X}
\toprule {\bf Date} & {\bf Version} & {\bf Notes}\\
\midrule
Date 1 & 1.0 & Notes\\
Date 2 & 1.1 & Notes\\
\bottomrule
\end{tabularx}
\end{table}
\newpage
\pagenumbering{arabic}
\section{Introduction}
Decomposing a system into modules is a commonly accepted approach to developing
software.  A module is a work assignment for a programmer or programming
team~\citep{ParnasEtAl1984}.  We advocate a decomposition
based on the principle of information hiding~\citep{Parnas1972a}.  This
principle supports design for change, because the ``secrets'' that each module
hides represent likely future changes.  Design for change is valuable in SC,
where modifications are frequent, especially during initial development as the
solution space is explored.  
Our design follows the rules layed out by \citet{ParnasEtAl1984}, as follows:
\begin{itemize}
\item System details that are likely to change independently should be the
  secrets of separate modules.
\item Each data structure is used in only one module.
\item Any other program that requires information stored in a module's data
  structures must obtain it by calling access programs belonging to that module.
\end{itemize}
After completing the first stage of the design, the Software Requirements
Specification (SRS), the Module Guide (MG) is developed~\citep{ParnasEtAl1984}. The MG
specifies the modular structure of the system and is intended to allow both
designers and maintainers to easily identify the parts of the software.  The
potential readers of this document are as follows:
\begin{itemize}
\item New project members: This document can be a guide for a new project member
  to easily understand the overall structure and quickly find the
  relevant modules they are searching for.
\item Maintainers: The hierarchical structure of the module guide improves the
  maintainers' understanding when they need to make changes to the system. It is
  important for a maintainer to update the relevant sections of the document
  after changes have been made.
\item Designers: Once the module guide has been written, it can be used to
  check for consistency, feasibility and flexibility. Designers can verify the
  system in various ways, such as consistency among modules, feasibility of the
  decomposition, and flexibility of the design.
\end{itemize}
The rest of the document is organized as follows. Section
\ref{SecChange} lists the anticipated and unlikely changes of the software
requirements. Section \ref{SecMH} summarizes the module decomposition that
was constructed according to the likely changes. Section \ref{SecConnection}
specifies the connections between the software requirements and the
modules. Section \ref{SecMD} gives a detailed description of the
modules. Section \ref{SecTM} includes two traceability matrices. One checks
the completeness of the design against the requirements provided in the SRS. The
other shows the relation between anticipated changes and the modules. Section
\ref{SecUse} describes the use relation between modules.
\section{Anticipated and Unlikely Changes} \label{SecChange}
This section lists possible changes to the system. According to the likeliness
of the change, the possible changes are classified into two
categories. Anticipated changes are listed in Section \ref{SecAchange}, and
unlikely changes are listed in Section \ref{SecUchange}.
\subsection{Anticipated Changes} \label{SecAchange}
Anticipated changes are the source of the information that is to be hidden
inside the modules. Ideally, changing one of the anticipated changes will only
require changing the one module that hides the associated decision. The approach
adapted here is called design for
change.

Due to API compatability issues any code that relies on an outside API is maintained in its own module. This means particular social media API's need to be kept hidden from one another so if there is modifications that affect the original it will not affect the application as a whole. This means there will be two replacement modules:

Facebook feed replacer.
Replacer for all other websites.

Additionally the image replacement software will need to be kept seperate due to relying on API's. The image replacer relies on GIPHY RS feeds to obtain images. Image recognition software for identification of memes also replies on Google's Vision API and will be needed to be kept modular from the rest of the modules. 
\begin{description}
\item[\refstepcounter{acnum} \actheacnum \label{acHardware}:] The specific
  hardware on which the software is running.

There is minimal hardware required for Rather to run. It requires a laptop that is able to access Google Chrome. There are minimal input devices required. Some form of text communicator is required. This could be done through a keyboard or a speech to text convertor. Additionally some sort of selction input device is required such as a mouse. These are unlikely to change drastically anytime soon and will not accounted for in the design 
\item[\refstepcounter{acnum} \actheacnum \label{acInput}:] The format of the
  initial input data.
\item The input data is sourced from the internet and user input. 

User input:
The user is required to input items to filter, and a desired replacement. This is done through strings that are communicated textually.

Facebook: There is a variety of input required from Facebook that the user interacts with. There is the source code that will be obtained from each individual feed used. Additionally due to webpages being loaded dynamically, frequent reloading of the source code is required so the application will continue to interact with the newsfeed. Some webpages have unique streamlet like information that will replace the entire webpage rather than the individual feed. The actual streamlet information will need to be uniquely obtained. This will be done for Facebook specifically. These each have APIs that have been previously developed and supported to give this information.  
\end{description}
\subsection{Unlikely Changes} \label{SecUchange}
The module design should be as general as possible. However, a general system is
more complex. Sometimes this complexity is not necessary. Fixing some design
decisions at the system architecture stage can simplify the software design. If
these decision should later need to be changed, then many parts of the design
will potentially need to be modified. Hence, it is not intended that these
decisions will be changed.
\begin{description}
\item[\refstepcounter{ucnum} \uctheucnum \label{ucIO}:] Input/Output devices
  (Input: File and/or Keyboard, Output: File, Memory, and/or Screen).

The input will always be done textually. This hardware input method is unlikely to change anytime soon.

The output will be done through a screen and internet. These are pieces of hardware that are unlikely to be modified anytime soon. It is assumed there will always be a visual output for the user, and that webpages will always contain source code.
\item[\refstepcounter{ucnum} \uctheucnum \label{ucInput}:] There will always be
  a source of input data external to the software.
\item ... % what's this?
\end{description}
\section{Module Hierarchy} \label{SecMH}
This section provides an overview of the module design. Modules are summarized
in a hierarchy decomposed by secrets in Table 1. The modules listed
below, which are leaves in the hierarchy tree, are the modules that will
actually be implemented.

Due to the nature of javascript code, modules are not often used. We used modules by having seperate javascript files that performed each action. These wll be treated as modules in the tables
\begin{description}
\item [\refstepcounter{mnum} \mthemnum \label{mHH}:] Hardware-Hiding Module
\item There are no hardware hiding modules on this project. The hardware will not affect how the code operates provided that there is something that matches the input and output style. 
\end{description}
\begin{table}[h!]
\centering
\begin{tabular}{p{0.3\textwidth} p{0.6\textwidth}}
\toprule
\textbf{Level 1} & \textbf{Level 2}\\
\midrule
{Hardware-Hiding Module} & ~ \\
\midrule
\multirow{6}{0.3\textwidth}{Behaviour-Hiding Module}\\ % why the question mark?
& Facebook-API.js\\
&Filter-Feature.js\\
&Detect-Website.js\\
& page.js\\
& popup.js\\
& content.js\\
\midrule
% javascript doesn't have a hyphen in it
\multirow{2}{0.3\textwidth}{Non Javascript Modules and Design modules}\\ % why the question mark?
& manifest.json\\
& popup.html \\
\bottomrule
\end{tabular}
\caption{Module Hierarchy}
\label{TblMH}
\end{table}
\section{Connection Between Requirements and Design} \label{SecConnection}
The design of the system is intended to satisfy the requirements developed in
the SRS. In this stage, the system is decomposed into modules. The connection
between requirements and modules is listed in Table 3.

\begin{table}[h!]
\centering
\begin{tabular}{p{0.3\textwidth} p{0.6\textwidth}}
\toprule
\textbf{Level 1} & \textbf{Level 2}\\
\midrule
{Module and requirements} & ~ \\
\midrule
\multirow{7}{0.3\textwidth}{Functional}\\
& Facebook-API.js ----- Allow the user to block specific items in their newsfeed\\
&Filter-Feature.js ----- Allow the user to filter individual post they do not want censored\\
&Detect-Website.js ----- Allow the user to block items on specific web feeds. (Will allow functionality for Twitter and Facebook)\\
&Image.js -----If a keyword is found images containing or tagged with it will be filtered\\
& page.js ----- Main Controller for direction of information flow and program state\\
& popup.js ----- Allow the user to filter content on websites that are not twitter and facebook\\
& content.js ----- Allow the user to detect both content and tags related to a word\\
& popup.html ----- Allow the user to input a "kill list" of words they want to remove\\
& popup.html ----- Allow the user to input a "replacementl list" of words they want to remove\\
& popup.html ----- Allow the user an option to mute or replace content\\
\midrule
\multirow{3}{0.3\textwidth}{Non-Functional} & {?}\\
& manifest.json ----- performance requirements and file naming index file\\
& popup.html ----- Allow the program to have a sleek look and feel\\

\bottomrule
\end{tabular}
\caption{The modules that satisfies each requirement is listed}
\label{TblMH}
\end{table}


\section{Module Decomposition} \label{SecMD}
Modules are decomposed according to the principle of ``information hiding''
proposed by \citet{ParnasEtAl1984}. The \emph{Secrets} field in a module
decomposition is a brief statement of the design decision hidden by the
module. The \emph{Services} field specifies \emph{what} the module will do
without documenting \emph{how} to do it. For each module, a suggestion for the
implementing software is given under the \emph{Implemented By} title. If the
entry is \emph{OS}, this means that the module is provided by the operating
system or by standard programming language libraries.  Also indicate if the
module will be implemented specifically for the software.
Only the leaf modules in the
hierarchy have to be implemented. If a dash (\emph{--}) is shown, this means
that the module is not a leaf and will not have to be implemented. Whether or
not this module is implemented depends on the programming language
selected.
\subsection{Hardware Hiding Modules (\mref{mHH})}
\begin{description}
\item[Secrets:]The data structure and algorithm used to implement the virtual
  hardware.
\item[Services:]Serves as a virtual hardware used by the rest of the
  system. This module provides the interface between the hardware and the
  software. So, the system can use it to display outputs or to accept inputs.
\item[Implemented By:] OS
\end{description}
\subsection{Behaviour-Hiding Module}
\begin{description}
\item[Secrets:]The contents of the required behaviours.
\item[Services:]Includes programs that provide externally visible behaviour of
  the system as specified in the software requirements specification (SRS)
  documents. This module serves as a communication layer between the
  hardware-hiding module and the software decision module. The programs in this
  module will need to change if there are changes in the SRS.
\item[Implemented By:] Facebook-API.js, Twitter-API.js, Filter-Feauture.js, content.js, popup.html
\end{description}
\subsubsection{Input Format Module} %% this doesn't get rendered correctly
\begin{description}
\item[Secrets:]The format and structure of the input data.
\item[Services:]Converts the input data into the data structure used by the
  input parameters module.
\item[Implemented By:] popup.html, content.js
\end{description}
\subsection{Software Decision Module}
\begin{description}
\item[Secrets:] The design decision based on mathematical theorems, physical
  facts, or programming considerations. The secrets of this module are
  \emph{not} described in the SRS.
\item[Services:] Includes data structure and algorithms used in the system that
  do not provide direct interaction with the user. 
  % Changes in these modules are more likely to be motivated by a desire to
  % improve performance than by externally imposed changes.
\item[Implemented By:] Detect-Website.js, page.js
\end{description}
\section{Traceability Matrix} \label{SecTM}
This section shows two traceability matrices: between the modules and the
requirements and between the modules and the anticipated changes.
% the table should use mref, the requirements should be named, use something
% like fref
\begin{table}[H]
\centering
\begin{tabular}{p{0.6\textwidth} p{0.2\textwidth}}
\toprule
\textbf{Req.} & \textbf{Modules}\\
\midrule
1. User can input "kill list" &  popup.html\\
2. User can input "replacement list" &  popup.html\\
3. User can decide whether to mute or replace posts &  popup.html\\
4. Program identifies individual posts and is only active on Facebook. &  Facebook-API.js, Detect-Website.js\\ 
5. Program searches for keywords in both text and image contents.  &  Image.js, Content.js\\
6. Program can identify images within a post. &  Image.js\\
7. Program mutes or replaces content when keywords are found. &  Image.js, Content.js, Popup.js\\
8. Program can "un-filter" specific people's posts. &  popup.html\\
\bottomrule
\end{tabular}
\caption{Trace Between Requirements and Modules}
\label{TblRT}
\end{table}
\begin{table}[H]
\centering
\begin{tabular}{p{0.6\textwidth} p{0.2\textwidth}}
\toprule
\textbf{AC} & \textbf{Modules}\\
\midrule
Facebook-API changes & Facebook-API.js\\
Input style changes & popup.html\\
Webpage does not load completely initially & Detect-Website.js\\
\bottomrule
\end{tabular}
\caption{Trace Between Anticipated Changes and Modules}
\label{TblACT}
\end{table}
\section{Use Hierarchy Between Modules} \label{SecUse}
In this section, the uses hierarchy between modules is
provided. \citet{Parnas1978} said of two programs A and B that A {\em uses} B if
correct execution of B may be necessary for A to complete the task described in
its specification. That is, A {\em uses} B if there exist situations in which
the correct functioning of A depends upon the availability of a correct
implementation of B.  Figure \ref{FigUH} illustrates the use relation between
the modules. It can be seen that the graph is a directed acyclic graph
(DAG). Each level of the hierarchy offers a testable and usable subset of the
system, and modules in the higher level of the hierarchy are essentially simpler
because they use modules from the lower levels.
\begin{figure}[H]
\centering
\includegraphics[width=0.7\textwidth]{UsesHierarchy.png}
\caption{Use hierarchy among modules}
\label{FigUH}
\end{figure}
%\section*{References}
\bibliographystyle {plainnat}
\bibliography {MG}
\end{document}