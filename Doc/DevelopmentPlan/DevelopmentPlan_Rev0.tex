\documentclass{article}
\usepackage{booktabs}
\usepackage{tabularx}
\title{SE 3XA3: Development Plan\\Rather}
\author{Team 3
		\\ Erin Varey, vareye
		\\ Nik Novak, novakn
		\\ Joel Straatman, straatjc
}
\date{}
%% Comments

\usepackage{color}

\newif\ifcomments\commentstrue

\ifcomments
\newcommand{\authornote}[3]{\textcolor{#1}{[#3 ---#2]}}
\newcommand{\todo}[1]{\textcolor{red}{[TODO: #1]}}
\else
\newcommand{\authornote}[3]{}
\newcommand{\todo}[1]{}
\fi

\newcommand{\wss}[1]{\authornote{blue}{SS}{#1}}
\newcommand{\ds}[1]{\authornote{red}{DS}{#1}}
\newcommand{\mj}[1]{\authornote{red}{MSN}{#1}}
\newcommand{\cm}[1]{\authornote{red}{CM}{#1}}
\newcommand{\mh}[1]{\authornote{red}{MH}{#1}}

% team members should be added for each team, like the following
% all comments left by the TAs or the instructor should be addressed
% by a corresponding comment from the Team

\newcommand{\tm}[1]{\authornote{magenta}{Team}{#1}}

\begin{document}
\begin{table}[hp]
\caption{Revision History} \label{TblRevisionHistory}
\begin{tabularx}{\textwidth}{llX}
\toprule
\textbf{Date} & \textbf{Developer(s)} & \textbf{Change}\\
\midrule
September 29 2016 & all & Created Development Plan\\

\bottomrule
\end{tabularx}
\end{table}
\newpage
\maketitle
This development plan is a detailed structure of how we will redevelop Rather, an open source Chrome extension for replacing text.
\section{Team Meeting Plan}
The team is going to meet twice a week on Tuesday and Thursday evening's for a maximum of 30 minutes. Every team member is expected to bring an agenda with what they have accomplished and any issues they have encountered. Each section of code developed will be passed on to the tester Quality Assurance developer for testing. Any member who misses a meeting without a valid excuse needs to buy the group a coffee. The meetings will be adjourned with a work plan of what everyone is expected to do until the next meeting. This will allow much of the coding to be done on each members time to ensure a quality job is done. 
\section{Team Communication Plan}
The group will communicate through a variety of online methods. We will use Git to communicate code updates, Facebook to set up meetings, Skype for online meetings, and Google Docs to work on documentation before it is translated to LaTeX.
\section{Team Member Roles}
\subsection{Nik Novak- Lead developer}
The role of the lead developer is to control the code segments as they are developed. Each member will develop various sections of the code but the role of the lead developer is to oversee each piece. They will assign sections to various members during the development process. They will also oversee each section to ensure it is being held to coding standard and style. 
\subsection{Erin Varey - Quality Assurance developer}
The QA developer will be the predominant tester of each code segment. As sections are completed the QA will thoroughly test each section to ensure it is robust and functional. When the QA develops sections of code the lead developer will be the QA tester. 
\subsection{Joel Straatman- Scrum Master}
The group will attempt to follow the scrum structure. The scrum master will over see the whole process and keep up to date with various due dates. The scrum master will ensure the team follows the outline of goals as described in the Gant chart and modify it as needed. 
\section{Git Workflow Plan}
Within our git hierarchy we will have it split into code and documentation. Every team member will pull the current code before working and then commit their changes every couple hours. Within our documentation each required document will get its own folder. The code will be broken up into functions that will be committed individually to make it easier to test and modularize. More subfolders in each branch will be created as needed. The repo happened to be created by Nik Novak, but will be maintained by each group member as equally as possible.
\section{Proof of Concept Demonstration Plan}
The group has some experience with Javascript and has developed a basic test plan that the Quality Assurance developer will implement after every change. With regular meetings this code should be able to be completed in a reasonable timeframe. The scope of the project is small enough such that we can complete it and add additional features to continue to challenge ourselves. Each team member has previously completed small personal projects written in Javascript, with one having produced a website relying heavily on Javascript. Additional each member is well versed in how to use %% capitalize proper nouns
google chrome extensions once successfully completed. 
\section{Technology}
This will be coded in Javascript and CSS using the IDE of choice. Every group member has a computer to work on and commit to git with.
\section{Coding Style}
The group will model our coding style after Google; tabs for spacing code, constants written in all caps. The code will follow information hiding to ensure it is well designed. The group will test both blackbox and whitebox based on a test plan created by the Quality Assurance developer. Any scripting used for automation of testing will be written in nano.
\section{Project Schedule}
The Gant Chart is located in the git repository under Master/doc/Development Plan.
\section{Project Review}
This project will be a good test of not only our skills in languages we have not formally learned yet, but also our ability to use our coding skills to create something usable in the real world. Up until now, a majority of software engineering students only have experience with creating something that runs in an IDE. This project provides us with the interesting challenge of building upon our basic coding skills to create something that we and other people can use in their everyday lives. 

\end{document}