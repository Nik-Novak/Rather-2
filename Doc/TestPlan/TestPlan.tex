\documentclass[12pt, titlepage]{article}
\usepackage{booktabs}
\usepackage{tabularx}
\usepackage{hyperref}
\hypersetup{
    colorlinks,
    citecolor=black,
    filecolor=black,
    linkcolor=red,
    urlcolor=blue
}
\usepackage[round]{natbib}
\title{SE 3XA3: Test Plan\\Title of Project}
\author{Team \# 3, Team 3
		\\ Erin Varey Vareye
		\\ Joel Straatman Straatjc
		\\ Nik Novak 	Novakn
}
\date{\today}
%% Comments

\usepackage{color}

\newif\ifcomments\commentstrue

\ifcomments
\newcommand{\authornote}[3]{\textcolor{#1}{[#3 ---#2]}}
\newcommand{\todo}[1]{\textcolor{red}{[TODO: #1]}}
\else
\newcommand{\authornote}[3]{}
\newcommand{\todo}[1]{}
\fi

\newcommand{\wss}[1]{\authornote{blue}{SS}{#1}}
\newcommand{\ds}[1]{\authornote{red}{DS}{#1}}
\newcommand{\mj}[1]{\authornote{red}{MSN}{#1}}
\newcommand{\cm}[1]{\authornote{red}{CM}{#1}}
\newcommand{\mh}[1]{\authornote{red}{MH}{#1}}

% team members should be added for each team, like the following
% all comments left by the TAs or the instructor should be addressed
% by a corresponding comment from the Team

\newcommand{\tm}[1]{\authornote{magenta}{Team}{#1}}

\begin{document}
\maketitle
\pagenumbering{roman}
\tableofcontents
\listoftables
\listoffigures
\begin{table}[bp]
\caption{\bf Revision History}
\begin{tabularx}{\textwidth}{p{3cm}p{2cm}X}
\toprule {\bf Date} & {\bf Version} & {\bf Notes}\\
\midrule
October 21 & 1.0 & Started to create\\
Date 2 & 1.1 & Notes\\
\bottomrule
\end{tabularx}
\end{table}
\newpage
\pagenumbering{arabic}
This document ...
\section{General Information}
\subsection{Purpose}
\subsection{Scope}
\subsection{Acronyms, Abbreviations, and Symbols}
	
\begin{table}[hbp]
\caption{\textbf{Table of Abbreviations}} \label{Table}
\begin{tabularx}{\textwidth}{p{3cm}X}
\toprule
\textbf{Abbreviation} & \textbf{Definition} \\
\midrule
Abbreviation1 & Definition1\\
Abbreviation2 & Definition2\\
\bottomrule
\end{tabularx}
\end{table}
\begin{table}[!htbp]
\caption{\textbf{Table of Definitions}} \label{Table}
\begin{tabularx}{\textwidth}{p{3cm}X}
\toprule
\textbf{Term} & \textbf{Definition}\\
\midrule
Term1 & Definition1\\
Term2 & Definition2\\
\bottomrule
\end{tabularx}
\end{table}	
\subsection{Overview of Document}
\section{Plan}
	
\subsection{Software Description}
\subsection{Test Team}
The Software will best tested by the developers along with a small focus group. The small focus group will consist of 3 volunteers who will test the code to prevent developer bias. 
\subsection{Automated Testing Approach}
Due to the nature of a web application the automated testing approach is going to be slightly complicated. It will consist of two components. A Java application for automation, and a Javascript script for testing the functions as required. The Java portion of the code will consist of automation that opens a Chrome browser and opens a webapge that will be used for testing. The program will then contorl the mouse to click the application and add a filter and then refresh the page. It will then execute the Javascript portion. This will consist of the unit testing scripts that will be written in Q-unit. 
\subsection{Testing Tools}
The only tool that is required for testing the application is a computer and a internet connection.
\subsection{Testing Schedule}
		
See Gantt Chart at the following url ...
\section{System Test Description}
	
\subsection{Tests for Functional Requirements}
\subsubsection{Search Functionality}
		
\paragraph{Can the program search and replace text?}
\begin{enumerate}
\item{test-id1\\}
Type: Functional, Dynamic, Manual, Static etc.*************
					
Initial State: Fully Loaded Dog Wikipedia Page
					
Input: Words to block: Dog, Canine
Words to replace with: Elephant, Rooster
					
Output: The Wikipedia page will be reloaded with all the blocked words removed and randomly replaced with either Elephant or Rooster.
					
How test will be performed: The input will be inserted through the GUI for the web application. Then a search will be performed on the page for Dog and Canine to ensure the word doesn't occur anywhere
					
\item{test-id2\\}
Type: Functional, Dynamic, Manual, Static etc.*******************
					
Initial State: A pre-made Twitter page that will be used for testing. The page will consist of four tweets. The link is attached below:

As seen this page contains a series of Tweets with various key words that will be used for testing.
					
Input:  Words to block: Donald Trump
Words to replace with: Puppies
					
Output: The page will be reloaded and all strings that orginally contained Donald Trump will have the word Puppies instead. 

How test will be performed: The input will be inserted through the GUI for the web application. Then a search will be performed on the page for Donald Trump to ensure the phrase doesn't occur anywhere.
\end{enumerate}
\subsubsection{Image Replacement}
\item{test-id1\\}
Type: Functional, Dynamic, Manual, Static etc. *******************
					
Initial State: Fully Loaded Dog Wikipedia Page
					
Input: Words to block: Dog, Canine
Words to replace with: Elephant, Rooster
					
Output: The Wikipedia page will be reloaded with all the dog images removed and replace with an image tagged with Elephant or Rooster.
					
How test will be performed: The input will be inserted through the GUI for the web application. It will search the <img> tags for the new source code and pass if the image has been replaced with one of the tagged words. 
					
\item{test-id2\\}
Type: Functional, Dynamic, Manual, Static etc. *******************************
					
Initial State: A pre-made Twitter page that will be used for testing. The page will consist of  a series tweets involving commonly used key words. The link is attached below:

As seen this page contains a series of Tweets with various key words that will be used for testing.
					
Input:  Words to block: Donald Trump
Words to replace with: Puppies
					
Output: The page will be reloaded and all Tweets that contain and image of Donald Trump will be replaced with a picture of a puppy.

How test will be performed: The input will be inserted through the GUI for the web application. It will search the <img> tags for the new source code and pass if the image has been replaced with one of the tagged words. 
\end{enumerate}
\subsubsection{Selective User Filtering}
\item{test-id1\\}
Type: Functional, Dynamic, Manual, Static etc. ***********************
					
Initial State: Fully Loaded Dog Wikipedia Page
					
Input: Words to block: Dog, Canine
Words to replace with: Elephant, Rooster
Exceptions?: 3XA3TEST
					
Output: The Wikipedia page will be reloaded with all the dog images removed and replace with an image tagged with Elephant or Rooster. Since Wikipedia is not a social media website selective filtering does not apply. No additional changes should occur with this test case.
					
How test will be performed: The input will be inserted through the GUI for the web application. It will search the page to ensure the exception has not been removed if it existed in the original source code.
					
\item{test-id2\\}
Type: Functional, Dynamic, Manual, Static etc.
					
Initial State: A pre-made Twitter page that will be used for testing. The page will consist of  a series tweets involving commonly used key words. The link is attached below:

As seen this page contains a series of Tweets with various key words that will be used for testing.
					
Input:  Words to block: Donald Trump
Words to replace with: Puppies
Exceptions?: 3XA3TEST
					
Output: The page will be reloaded and all the filtering will occur as expected. However any tweets that were created by the user listed under exceptions will not be filtered.

How test will be performed: The input will be inserted through the GUI for the web application. It will search the text and <img> tags for the filtering. Any tweets from the exception user will NOT be removed in the modified source code. A search will be performed comparing the source code to ensure this occurred.
\end{enumerate}
\subsection{Tests for Nonfunctional Requirements}
\subsubsection{Application Security Testing}
\item{test-id1\\}
*******JOEL SECTION
Type: Functional, Dynamic, Manual, Static etc. ***********************
					
Initial State: Fully Loaded Dog Wikipedia Page
					
Input: Words to block: Dog, Canine
Words to replace with: Elephant, Rooster
Exceptions?: 3XA3TEST
					
Output: The Wikipedia page will be reloaded with all the dog images removed and replace with an image tagged with Elephant or Rooster. Since Wikipedia is not a social media website selective filtering does not apply. No additional changes should occur with this test case.
					
How test will be performed: The input will be inserted through the GUI for the web application. It will search the page to ensure the exception has not been removed if it existed in the original source code.
					
\item{test-id2\\}
Type: Functional, Dynamic, Manual, Static etc.
					
Initial State: A pre-made Twitter page that will be used for testing. The page will consist of  a series tweets involving commonly used key words. The link is attached below:

As seen this page contains a series of Tweets with various key words that will be used for testing.
					
Input:  Words to block: Donald Trump
Words to replace with: Puppies
Exceptions?: 3XA3TEST
					
Output: The page will be reloaded and all the filtering will occur as expected. However any tweets that were created by the user listed under exceptions will not be filtered.

How test will be performed: The input will be inserted through the GUI for the web application. It will search the text and <img> tags for the filtering. Any tweets from the exception user will NOT be removed in the modified source code. A search will be performed comparing the source code to ensure this occurred.
\end{enumerate}
\subsubsection{Response Time}
		
\paragraph{Does the program take to long to execute?}
\begin{enumerate}
\item{test-id1\\}
Type: ********* Due to the nature of internet response time the program is design to execute in under five seconds. However if someone is using bad dial up internet this response time may be extended due to bad internet. Therefore this test cannot be automated without knowing the users speed so it must be tested by the focus group using a timer.
					
Initial State: Wikipedia page for Dogs
					
Input/Condition: Input: Words to block: Dog, Canine
Words to replace with: Elephant, Rooster
Exceptions?: 3XA3TEST
					
Output/Result: The page will reload in under 5 seconds with the removed words
					
How test will be performed: This test cannot be automated due to its subjective nature. The focus group members will rate this pass or fail given the timer they user. This will allow them to make exceptions for poor connection.
					
\item{test-id2\\}
Type: Functional, Dynamic, Manual, Static etc.
					
Initial State: The test Twitter page
					
Input:
Input:  Words to block: Donald Trump
Words to replace with: Puppies
Exceptions?: 3XA3TEST
					
Output: The replacement will occur in under 5 seconds when timed by the user.
					
How test will be performed: A focus group member will manually verify this test case.
\end{enumerate}
\subsubsection{Visual Appeal}
\item{test-id1\\}
Due to the subjective nature of this requirement this requirement isn't automatable. This will be done with the focus group.

Type:
Initial State: Wikipedia page for Dogs
					
Input/Condition: Input: Words to block: Dog, Canine
Words to replace with: Elephant, Rooster
Exceptions?: 3XA3TEST
					
Output/Result: All of the focus group members will give the visual at least a 7/10 in order for this test case to pass.
					
How test will be performed: The focus group memeber will perform the test. They will then rank the reloaded page out of 10.
\end{enumerate}
\item{test-id2\\}
Due to the subjective nature of this requirement this requirement isn't automatable. This will be done with the focus group.

Type:
Initial State: The GUI window will be opened for rating.
					
Input/Condition: No user input, just open the web application
					
Output/Result: All of the focus group members will give the visual at least a 7/10 in order for this test case to pass.
					
How test will be performed: The focus group memeber will perform the test. They will then rank the reloaded page out of 10.
\end{enumerate}
\subsubsection{Failure Response Testingl}
\item{test-id1\\}
The program should be able to encounter errors without crashing the user's browser. Error Catching will be covered here.

Type:
Initial State: Wikipedia page for Dogs
					
Input/Condition: Input: Words to block: .*
Words to replace with: Elephant, Rooster
Exceptions?: 
					
Output/Result: The regular expression will replace everything on the page with one of the replacement words. This is a massive amount of replacement and can cause the page to crash. The program should be able to handle the new output without creating an error on the browser.
					
How test will be performed: This is difficult to decifer using automation so the focus group will have to be used. The actual performing of the test can be automated but someone will need to visually verify if the page crashed.
\end{enumerate}
\item{test-id2\\}
Type:
Initial State: Test Twitter Page
					
Input/Condition: Words to block: .*
Words to replace with: Elephant, Rooster
Exceptions?: 
					
Output/Result: Everything on the page should be replaced with the words or tagged images without the page crashing. Response time will also be noted in this as serious lag is also considered crashing.
					
How test will be performed: The occurance can be automated but the yes no of the page crashing needs to be noted subjectively by the focus group.
\end{enumerate}
\section{Tests for Proof of Concept}
\subsection{Area of Testing1}
		
\paragraph{Title for Test}
\begin{enumerate}
\item{test-id1\\}
Type: Functional, Dynamic, Manual, Static etc.
					
Initial State: 
					
Input: 
					
Output: 
					
How test will be performed: 
					
\item{test-id2\\}
Type: Functional, Dynamic, Manual, Static etc.
					
Initial State: 
					
Input: 
					
Output: 
					
How test will be performed: 
\end{enumerate}
\subsection{Area of Testing2}
...
	
\section{Comparison to Existing Implementation}	
				
\section{Unit Testing Plan}
		
\subsection{Unit testing of internal functions}
		
\subsection{Unit testing of output files}		
\bibliographystyle{plainnat}
\bibliography{SRS}
\newpage
\section{Appendix}
This is where you can place additional information.
\subsection{Symbolic Parameters}
The definition of the test cases will call for SYMBOLIC\_CONSTANTS.
Their values are defined in this section for easy maintenance.
\subsection{Usability Survey Questions?}
This is a section that would be appropriate for some teams.
\end{document}