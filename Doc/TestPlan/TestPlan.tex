\documentclass[12pt, titlepage]{article}
\usepackage{booktabs}
\usepackage{tabularx}
\usepackage{hyperref}
\hypersetup{
    colorlinks,
    citecolor=black,
    filecolor=black,
    linkcolor=red,
    urlcolor=blue
}
\usepackage[round]{natbib}
\title{SE 3XA3: Test Plan\\Title of Project}
\author{Team \# 3, Team 3
		\\ Erin Varey Vareye
		\\ Joel Straatman Straatjc
		\\ Nik Novak 	Novakn
}
\date{\today}
%% Comments

\usepackage{color}

\newif\ifcomments\commentstrue

\ifcomments
\newcommand{\authornote}[3]{\textcolor{#1}{[#3 ---#2]}}
\newcommand{\todo}[1]{\textcolor{red}{[TODO: #1]}}
\else
\newcommand{\authornote}[3]{}
\newcommand{\todo}[1]{}
\fi

\newcommand{\wss}[1]{\authornote{blue}{SS}{#1}}
\newcommand{\ds}[1]{\authornote{red}{DS}{#1}}
\newcommand{\mj}[1]{\authornote{red}{MSN}{#1}}
\newcommand{\cm}[1]{\authornote{red}{CM}{#1}}
\newcommand{\mh}[1]{\authornote{red}{MH}{#1}}

% team members should be added for each team, like the following
% all comments left by the TAs or the instructor should be addressed
% by a corresponding comment from the Team

\newcommand{\tm}[1]{\authornote{magenta}{Team}{#1}}

\begin{document}
\maketitle
\pagenumbering{roman}
\tableofcontents
\listoftables
\listoffigures
\begin{table}[bp]
\caption{\bf Revision History}
\begin{tabularx}{\textwidth}{p{3cm}p{2cm}X}
\toprule {\bf Date} & {\bf Version} & {\bf Notes}\\
\midrule
October 21 & 1.0 & Started to create\\
Date 2 & 1.1 & Notes\\
\bottomrule
\end{tabularx}
\end{table}
\newpage
\pagenumbering{arabic}
This document ...
\section{General Information}
\subsection{Purpose}
\subsection{Scope}
\subsection{Acronyms, Abbreviations, and Symbols}
	
\begin{table}[hbp]
\caption{\textbf{Table of Abbreviations}} \label{Table}
\begin{tabularx}{\textwidth}{p{3cm}X}
\toprule
\textbf{Abbreviation} & \textbf{Definition} \\
\midrule
Abbreviation1 & Definition1\\
Abbreviation2 & Definition2\\
\bottomrule
\end{tabularx}
\end{table}
\begin{table}[!htbp]
\caption{\textbf{Table of Definitions}} \label{Table}
\begin{tabularx}{\textwidth}{p{3cm}X}
\toprule
\textbf{Term} & \textbf{Definition}\\
\midrule
Term1 & Definition1\\
Term2 & Definition2\\
\bottomrule
\end{tabularx}
\end{table}	
\subsection{Overview of Document}
\section{Plan}
	
\subsection{Software Description}
\subsection{Test Team}
The Software will best tested by the developers along with a small focus group. The small focus group will consist of 3 volunteers who will test the code to prevent developer bias. 
\subsection{Automated Testing Approach}
Due to the nature of a web application the automated testing approach is going to be slightly complicated. It will consist of two components. A Java application for automation, and a Javascript script for testing the functions as required. The Java portion of the code will consist of automation that opens a Chrome browser and opens a webapge that will be used for testing. The program will then contorl the mouse to click the application and add a filter and then refresh the page. It will then execute the Javascript portion. This will consist of the unit testing scripts that will be written in Q-unit. 
\subsection{Testing Tools}
The only tool that is required for testing the application is a computer and a internet connection.
\subsection{Testing Schedule}
		
See Gantt Chart at the following url ...
\section{System Test Description}
	
\subsection{Tests for Functional Requirements}
\subsubsection{Search Functionality}
		
\paragraph{Can the program identify where the word occurs in all of the text?}
\begin{enumerate}
\item{test-id1\\}
Type: Functional, Dynamic, Manual, Static etc.
					
Initial State: Fully Loaded Dog Wikipedia Page
					
Input: Words to block: Dog, Canine
Words to replace with: Elephant, Rooster
					
Output: The Wikipedia page will be reloaded with all the blocked words removed and randomly replaced with either Elephant or Rooster.
					
How test will be performed: The input will be inserted through the GUI for the web application. Then a search will be performed on the page for Dog and Canine to ensure the word doesn't occur anywhere
					
\item{test-id2\\}
Type: Functional, Dynamic, Manual, Static etc.
					
Initial State: 
					
Input: 
					
Output: 
					
How test will be performed: 
\end{enumerate}
\subsubsection{Area of Testing2}
...
\subsection{Tests for Nonfunctional Requirements}
\subsubsection{Area of Testing1}
		
\paragraph{Title for Test}
\begin{enumerate}
\item{test-id1\\}
Type: 
					
Initial State: 
					
Input/Condition: 
					
Output/Result: 
					
How test will be performed: 
					
\item{test-id2\\}
Type: Functional, Dynamic, Manual, Static etc.
					
Initial State: 
					
Input: 
					
Output: 
					
How test will be performed: 
\end{enumerate}
\subsubsection{Area of Testing2}
...
\section{Tests for Proof of Concept}
\subsection{Area of Testing1}
		
\paragraph{Title for Test}
\begin{enumerate}
\item{test-id1\\}
Type: Functional, Dynamic, Manual, Static etc.
					
Initial State: 
					
Input: 
					
Output: 
					
How test will be performed: 
					
\item{test-id2\\}
Type: Functional, Dynamic, Manual, Static etc.
					
Initial State: 
					
Input: 
					
Output: 
					
How test will be performed: 
\end{enumerate}
\subsection{Area of Testing2}
...
	
\section{Comparison to Existing Implementation}	
				
\section{Unit Testing Plan}
		
\subsection{Unit testing of internal functions}
		
\subsection{Unit testing of output files}		
\bibliographystyle{plainnat}
\bibliography{SRS}
\newpage
\section{Appendix}
This is where you can place additional information.
\subsection{Symbolic Parameters}
The definition of the test cases will call for SYMBOLIC\_CONSTANTS.
Their values are defined in this section for easy maintenance.
\subsection{Usability Survey Questions?}
This is a section that would be appropriate for some teams.
\end{document}